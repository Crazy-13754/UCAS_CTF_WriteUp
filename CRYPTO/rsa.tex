\documentclass[a4paper]{article}  % 设置纸张大小
\usepackage[margin=1in]{geometry} % 设置边距,符合Word设定
\usepackage{ctex} % 对中文支持的宏包
\usepackage{cite} % 引用
\usepackage{graphicx} % 插入图片所需要的宏包
\usepackage{lipsum} % 生成虚拟文本(Lorem Ipsum)的宏包,测试或者占位用。
\usepackage[hidelinks]{hyperref} % 取消超链接颜色,并且目录有超链接
\newcommand{\upcite}[1]{\textsuperscript{\textsuperscript{\cite{#1}}}} % 定义上标引用,套两层会让上标更“上”一点
\title{\heiti\zihao{2} RSA}
\author{\songti Crazy\_13754}
\date{\today}

\begin{document}
    \maketitle
\begin{abstract}
    写了些关于rsa的东西
\end{abstract}
\tableofcontents
\section[引言]{引言}

网上找到的博客质量参差不齐。实际上,写这篇的时候发现有论文写的很清楚了\upcite{chenchuanbo2006rsa}。此外维基百科也写的非常好,本来把它们丢上来就可以了。你问我为什么还要写这篇文章?请看这个图 \ref{fig:rsa1}:

\begin{figure}[htbp]
    % [h]当前位置。将图形放置在正文文本中给出该图形环境的地方。如果本页所剩的页面不够,这一参数将不起作用。
    % [t]顶部。将图形放置在页面的顶部。
    % [b]底部。将图形放置在页面的底部。
    % [p]浮动页。将图形放置在一只允许有浮动对象的页面上。
    \centering
    \includegraphics[width=1.77in,height=1.75in]{contents/rsa1.jpg}
    \caption{This is an inserted jpg graphic}
    \label{fig:rsa1}
    \end{figure}

如果你还是没有明白我想说什么,请再看看这个表 \ref{table:1}:

\begin{table}[h!]
    \centering
    \begin{tabular}{ c c c } 
        \hline
        对写奇奇怪怪东西的看法 & 可以理解 & 不可理喻 \\
        \hline
        支持 & 0.1\% & 0.0\% \\ 
        不支持 & 0.2\% & 99.7\% \\ 
        \hline
       \end{tabular}
    \caption{Table to test captions and labels}
    \label{table:1}
    \end{table}

如果(虽然几乎是当然的)你还是不理解,那就看看这些东西:

\begin{figure}[htbp]
	\centering
	\begin{minipage}{0.49\linewidth}
		\centering
		\includegraphics[width=0.9\linewidth]{contents/rsa2.jpg}
		\caption{tupian1}
		\label{tupian1}%文中引用该图片代号
	\end{minipage}
	\begin{minipage}{0.49\linewidth}
		\centering
		\includegraphics[width=0.9\linewidth]{contents/rsa3.jpg}
		\caption{tupian2}
		\label{tupian2}%文中引用该图片代号
	\end{minipage}
	%\qquad
	%让图片换行,
	
	\begin{minipage}{0.49\linewidth}
		\centering
		\includegraphics[width=0.9\linewidth]{contents/rsa4.jpg}
		\caption{tupian3}
		\label{tupian3}%文中引用该图片代号
	\end{minipage}
	\begin{minipage}{0.49\linewidth}
		\centering
		\includegraphics[width=0.9\linewidth]{contents/rsa5.jpg}
		\caption{tupian4}
		\label{tupian4}%文中引用该图片代号
	\end{minipage}
\end{figure}

\clearpage
这显然把事情弄得更糟。好吧,只是我在学\LaTeX,而你浪费了不少时间来看刚刚的内容。而且你接下来看到的东西也会几乎全是复制的。

公钥密码系统的观点是由Diffie和Hell man在1796年首次提出的,它是密码学发展史上具有里程碑意义的一件大事。与传统对称密码体制(即加、解密密钥相同)相比,
公钥系统使用两个密钥:加密密钥可以公开,称为公钥;解密密钥保密,为私钥。产生公钥体制的内在动力有两个:

(1) 传统对称体制下密钥的存储和分配问题;

(2) 消息鉴别问题,就是指用来检验消息来自于声称的来源并且没有被修改。

公钥体制的基础是陷门(单向函数),即某种实际处理过程的不可逆性。目前的公钥思想基于两种:一是依赖于大数的因数分解的困难性;二是依赖于求模p离散对数的困难性。RSA密码算法就是基于大数的因数分解的困难性。


\section{过程}

\section[awa]{证明}
rsa的证明。

\bibliography{contents/rsaref} % 导入bib
\bibliographystyle{plain} % 参考文献排版风格
\end{document}